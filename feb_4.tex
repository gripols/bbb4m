%\lesson{1}{4 February 2025 13:05}{Chapter 1}

\chapter{Introduction to BBB4M}
\setcounter{chapter}{1}

\section{What is Trade?}%

\begin{definition}[Business]
    The manufacturing and/or sale of goods and/or services
    to satisfy customer demands.
\end{definition}

Trade is the exchange of goods and services between individuals, businesses, 
or nations. Trade allows for specialization, greater efficiency, and access 
to a wider variety of goods and services than would be available domestically.

\subsection{Trade Balance}
If a country is importing more than it is exporting, it is running a 
\textbf{trade deficit}. If a country is exporting more than it is importing, 
it is running a \textbf{trade surplus}.

A trade deficit is not necessarily bad; it depends on the context. 
For example, Canada and the United States have a strong trading relationship, 
and while Canada imports many consumer and industrial goods from the U.S., 
it also exports significant resources like crude oil. This exchange benefits both economies.

\begin{definition}[Transaction]
    An exchange of things of value.
\end{definition}

\subsection{Types of Business Transactions}
\begin{enumerate}
	\item \textbf{Domestic Business}: Transactions occur within the borders of one country.
    \item \textbf{International Business}: Transactions occur between businesses in different countries.
\end{enumerate}

\begin{definition}[Domestic Business]
	A business that makes most of its transactions within the borders
	of the country in which it is based.
\end{definition}

\begin{definition}[International Business]
	The economic system of transactions conducted between
	businesses located in different countries.
\end{definition}

\subsection{Market Definitions}
\begin{definition}[Domestic Market]
	The customers of a business who live in the country where the business
	operates.
\end{definition}

\begin{definition}[Foreign Market]
	The customers of a business who live in a different part of the world where
    the business operates.
\end{definition}

\begin{enumerate}
    \item Own a retail or distribution outlet in another country
    \item Own a manufacturing plant in another country
    \item Export goods and services to businesses or consumers in another country
    \item Import goods and services from businesses in another country
    \item Invest in businesses located in another country
\end{enumerate}

\begin{definition}[Trading Partner]
	When a business in Canada develops a relationship with a business in another
	country, that country becomes a trading partner with Canada.
\end{definition}

	It is important to note that international trade occurs between businesses, not entire countries.

\subsection{Canada's Trade Dependence on the U.S.}
Canada has a strong economic relationship with the United States. As of now (Feb. 2025),
approximately 75\% of Canadian exports go to the U.S. This close trade relationship makes 
Canada highly dependent on American economic policies and market conditions.

If tariffs or trade restrictions were imposed, it could have significant economic 
consequences for businesses and consumers on both sides.

\pagebreak
