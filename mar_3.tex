\section{Trade Barriers}

\begin{definition}[Tariff]
Tariffs, the most common type of trade barrier, are taxes 
or duties put on imported GT's from another country to influence 
products or economics, raise revenues, or product competitive advantages.
\end{definition}

It is not the consumer that pays for the tariff itself. Rather,
the cost of the tariff is passed onto the conusmer.\\ 

Applied to Trump's tariffs plan between Canada and America:
\begin{itemize}
  \item American companies sell goods to Canadian companies.
  \item A tariff is imposed by the American government onto all imports and exports.
  \item Canadian businesses continue importing American goods, but pass the cost onto the consumer.
\end{itemize}

\subsection{Tariff Winners v. Losers (Pros and Cons)}
\textbf{Winners}
\begin{itemize}
  \item The domestic company collects additional taxes.
  \item Goods from local producers are more competitively priced.
  \item Workers in local companies keep employment.
\end{itemize}
\textbf{Losers}
\begin{itemize}
  \item Goods from foreign producers are now more expensive.
  \item The price of products go up and consumers are forced to pay higher prices.
  \item Workers in foreign companies become at risk.
\end{itemize}

A trade imbalance is not inherently good or bad.
With Canada and America, the latter has significant buying power.

\subsection{Domestic and International Issues with Tariffs}

\subsubsection{Domestic Deadweight Loss}
A tariff generally leads to a \textbf{deadweight loss} 
(an excess loss or burden above the amount actually 
paid in tax) as it decreases aggregate economic 
activity and incomes.

\subsubsection{International Disequilibrium}
A tariff would throw off an established market equilibrium. 
A surplus could occur, forcing sellers to sell at lower prices, 
but what would be guaranteed is a recession.

\subsubsection{Overall Economic Slowdown}
Global economic activity recesses, with less money being paid to 
the American government.

\subsection{Trade Barriers}
Canada is heavily dependent on the United States for their economy.
About a third of Canada's GDP is dependent on exports, 75\% of which 
goes to the United States.

\begin{definition}[Protectionism]
The theory or practice of government policy shielding domestic 
industries from international trade, often through trade barriers such as tariffs.
\end{definition}

\subsection{What Motivates Protectionism?}
\begin{itemize}
    \item Response to "dumping"
    \item Response to chronic trade gap
    \item Employment protection 
    \item Protect "fledgling" infant sectors
    \item Protect key/strategic industries
    \item Raise revenues for the government
    \item Response to a recession/low demand
\end{itemize}

Most important things to focus on are: \textbf{Response to chronic trade gap, 
employment protection, raise revenues for the government.}

\begin{definition}[Trade Embargo]
    A government-imposed ban on trade of a specific product or with a specific country, 
    often declared to pressure foreign governments to change their policies.
\end{definition}

Compared to a sanction, an embargo completely shuts down trade between a country. Russia as a good example.

\begin{definition}[Trade Sanctions]
    Economic action taken by a country to coerce another country to conform to an 
    international agreement or norms of conduct.
\end{definition}

North Korea -- No weapons, commodities, luxury goods, travel, or financial services in or to.

\begin{definition}[Trade Quota]
    Government-imposed limit on the quantity, or in exceptional cases the value, 
    of the goods or services that may be exported or imported over a specified period of time.
\end{definition}

