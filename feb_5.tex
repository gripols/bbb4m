%\lesson{2}{5 February 2025 13:10}{Chapter 1}

\textbf{TODO: Chapter 1 - February 5}
\begin{itemize}
    \item Go through and complete review package.
    \begin{itemize}
        \item Notes:
        \begin{itemize}
            \item Collect and organize all of your notes from the textbook, class slides, and class discussions, and create and commit to Git repo. These notes should cover all of the material that was discussed in Chapter 1.
            \item Notes should be formatted in a logical, easy-to-follow manner. Include summaries, diagrams, and helpful material.
        \end{itemize}
        \item Provide detailed definitions for each term and explain their significance in the context of international trade. Use exaples when possible to illustrate the concepts. (\textbf{See Classroom Document for Reference.})
        \item Answer the Questions
        \begin{itemize}
            \item Knowlegde -- 2
            \item Thinking -- 12
            \item Thinking -- 13
            \item Communication -- 20
            \item Application -- 23
        \end{itemize}
    \end{itemize}
    \item Create Anki deck with key terms (use document on Classroom provided) and upload.
\end{itemize}

\section{History of Canadian Trade}

\begin{itemize}
    \item Explorers from France and England landed in what is now Canada in the 1600s.
    \item Traded with the First Nations people, especially the Ojibwe and the Cree, for fur and food, then sent goods back to Europe.
    \item Success of international business led to the establishment of colonies and outposts in Canada, notably Hudson's Bay Company and North West Company.
\end{itemize}

\subsection{(Past?) Trade with Europe}

\begin{itemize}
    \item Trade grew quickly after permanent settlements were established in Canada in the 1700s.
    \item Demand for raw materials (beaver pelt, fish, lumber) grew in Europe, where manufacturing took place.
    \item England defeated France in the Seven Years' War, which led to Canada's reliance on England for finished goods.
    \item Many major cities were established near ports to facilitate trade (Toronto, Montreal, etc.).  
\end{itemize}

\subsection{Trade with the United States}

\begin{itemize}
    \item The US declared independence from the UK in the late 1700s.
    \item It needed to become self-reliant.
    \item The invention of the steam engine and the cotton gin helped the rapid growth of American industry.
    \item Canada supplied raw materials that were needed in the US.
    \item The US became Canada's largest trading partner, which holds true to this day.
\end{itemize}

\subsection{Trade with Asia}

\begin{itemize}
    \item Canada began trading with Japan after WWII.
    \item Japan became known for high-quality electronics and cars.
    \item China has more recently become a trading partner.
    \item Chinese-made products are inexpensive and popular with North American retailers.
\end{itemize}
 
(add bit on here from slide deck)
\subsection{Trade with Mexico}

\begin{itemize}
    \item Developed since signing NAFTA in 1993.
    \item NAFTA has since been replaced with CUSMA, an agreement that aims to improve trade relations and economic co-operation among the three countries by setting rules for things like tariffs, IP rights, labor laws, and environmental standards.
    \item Mexico has since become one of Canada's top 5 trading partners. 
\end{itemize}

\subsection{Trade with Emerging Markets}

\subsubsection{Trade with the Middle East}

\begin{itemize}
    \item Has traditionally centred on oil, but this commodity is not sustainable.
    \item Political instability has limited trade at times.
    \item Saudi Arabia remains Canada's largest trading partner in the Middle East.
    \item Oil and gas, machinery, and defence lead the way.
\end{itemize}

\subsubsection{Trade with India}

\begin{itemize}
    \item Population of almost 1.5 billion people.
    \item 66\% of the population are below the age of 35.
    \item It has become a major centre of outsourcing and manufacturing.
    \item A lack of infastructure and widespread corruption are persistent.
    \item Indian companies are aggressively expanding into international markets.
\end{itemize}

\subsubsection{Trade with Africa}

\begin{itemize}
    \item African imports to Canada are very low.
    \item Business opportunities at time are limited by unstable governments, lack of infastructure, and rural economies.
    \item Rich in primary resources (metals, lumber, etc.)
    \item Some countries, such as South Africa and Nigeria, are beginning to emerge as Canada's major trading partners.
\end{itemize}

\pagebreak
