\section{Currency Fluctuations}

\begin{definition}[Exchange Rate]
    The amount of one's currency in relation to the currency of another country.
\end{definition}

The Canadian dollar is often quoted against the US dollar, as both countries 
heavily depend on each other in trade.

\textbf{Winners of a High Canadian Dollar}
\begin{itemize}
    \item Importers (purchased goods are relatively cheaper)
    \item Canadian travellers
    \item Major league sports teams in Canada
\end{itemize}

\textbf{Losers of a Low Canadian Dollar}
\begin{itemize}
    \item Exporters (less demand for their products)
    \item Canadian tourism
\end{itemize}

\begin{definition}[Floating Rate]
    An exchange rate that is not fixed in relation to other currencies.
\end{definition}

The price at which currency with a floating rate is bought and sold fluctuates according to supply and demand.
Most advanced economies use floating rates. (USA, Euro, China, Canada, etc.)

\begin{definition}[Currency Revaluation]
    The \textbf{increase} in value of a currency because the demand for 
    that particular currency is greater than supply.
\end{definition}

\pagebreak

\begin{definition}[Currency Devaluation]
    The \textbf{decrease} in value of a currency because the supply of 
    that particular currency is greater than the demand for it.
\end{definition}

\subsection{Factors Affecting Exchange Rate}
\begin{itemize}
    \item Economic conditions in Canada -- Inflation rate, unemployment rate, GDP, interest rates, etc.
    \item Trading between companies -- The more favorable the terms of trade (comparison of exports to imports), the higher the currency exchange.
    \item Politics -- Political tension and instability or the threat of terrorism decreases the demand for a currency.
    \item Psychological factors -- Historical significance and stability change the way currencies are viewed.
\end{itemize}

\begin{definition}[Hard Currencies]
    Stable currencies, such as the Euro, USD, etc., are easily converted to other currencies on the world exchange markets.
\end{definition}

\begin{definition}[Soft Currencies]
    A currency belonging to a country an economy that is small, weak, or fluctuates often, and is difficult to convert, such as the Russian Ruble.
\end{definition}