%\lesson{5}{28 February 2025}{Beginning Chapter 2}

\chapter{Trade in the Modern World}

\section{Foreign Portfolio Investment}
\begin{itemize}
    \item Investment in businesses located outside of Canada through stocks, bonds
    and other financial instruments. 
    \item Allows Canadians to spread out their investments, which is less risky than
    investing in just one area. 
    \item Also providess greater choice and opportunity.
\end{itemize}

Canada has a pretty small percentage of world economy, about 3\%. 
For the S\&P/TSX 60 Index, half of it is weighted in energy and finance. \\
Most of the western world is very conservative in global GDP growth,
as most of us are already there. 

\begin{definition}[Importing]
    To bring products or service into a country,
    for use by another business or for resale.
\end{definition}

Majority of the goods that Canada imports 
come from the United States. 

\begin{definition}[Global Sourcing]
    The process of a company buying equipment, 
    capital goods, raw materials, or services from around the world.
\end{definition}

\begin{definition}[Exporting]
    A good produced in one country that is sold into another 
    country. Seller of such good is an exporter; foreign buyers are importers.
\end{definition}

\begin{definition}[Value Adding]
    The amount of worth that is added to a product at 
    each stage of processing. It is the difference between 
    the cost of the raw materials and the finished goods.
\end{definition}

A value adding process could look like this:
Raw Material (50\$) \(\Rightarrow\) Manufacturing of Good (300\$) \(\Rightarrow\) Retail (500\$)